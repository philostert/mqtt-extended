\documentclass[fontsize=11pt,a4paper]{scrartcl}

\usepackage{graphicx}
\usepackage[T1]{fontenc}
\usepackage[ngerman]{babel}

\usepackage[tmargin=1in,bmargin=1in,lmargin=1.25in,rmargin=1.25in]{geometry}

\usepackage{ifxetex}

\ifxetex
\else
  \usepackage[utf8x]{inputenc}
\fi

\usepackage{mathptmx}
\usepackage[scaled=.92]{helvet}
\usepackage{courier}

\usepackage{listings}

\usepackage{microtype}

\usepackage{amsmath}
\usepackage{MnSymbol}
\usepackage{wasysym}

\newcommand{\HRule}{\rule{\linewidth}{0.2mm}}

\usepackage{hyperref}

\begin{document}

\setlength{\parindent}{0mm}

\begin{titlepage}

% Upper part of the page
\textsf{\large Technische Universität Berlin\\
Fachgebiet Telekommunikationsnetze (TKN)\\[0.5cm]}

\HRule \\[0.4cm]
\begin{minipage}{0.49\textwidth}
\begin{flushleft}
\textsf{Programmierpraktikum Kommunikationstechnologien}
\end{flushleft}
\end{minipage}
\begin{minipage}{0.49\textwidth}
\begin{flushright}
\textsf{Wintersemester 2015/16}
\end{flushright}
\end{minipage}
\\[0.4cm]
\HRule \\[1.0cm]

\today \\[0.5cm]

\textsf{{ \huge Gruppe Lokaler Broker \\[0.5cm] - Lastenheft - } \\[0.5cm]}

Daniel Happ \\
\textit{{\small \href{mailto:happ@tkn.tu-berlin.de?subject=Programmierpraktikum\%20Kommunikationstechnologien}{happ@tkn.tu-berlin.de}}} \\

\begin{center}
\vfill
\includegraphics[width=0.25\textwidth]{./images/TU_Logo_lang_1c_schwarz} \\[1.0cm]
\HRule
\end{center}

\end{titlepage}

\setlength{\parindent}{4mm}

\section{Zielbestimmung}
% Hier soll das Ziel des Projektes in eigenen Worten beschrieben werden.
% Dabei sollte auf die Bedeutung und den Nutzen des Projektergebnisses eingegangen werden.
% Dieser Abschnitt ist für alle Teams identisch, muss also nicht unbedingt angepasst werden.

Zu erstellen ist eine weitgehend autarke lokale Sensor-Plattform, die von der Messung von Sensorwerten über die Verbreitung von Sensorwerten im Netz bis hin zur Analyse und Visualisierung auf Endgeräten alle Funktionen lokal ohne externe Internet- bzw. Cloudverbindung bereitstellt.
Als Grundlage dient die Sensor-Plattform von relayr, die im Wesentlichen aus dem Sensor Starter Kit "Wunderbar" und der dazugehörigen Cloud-Platform besteht.

% Zielbestimmung für individuelles Team

Ziel der Gruppenarbeit ist es, den Nachrichtenaustausch zwischen den Komponenten der anderen Gruppen lokal zu gewährleisten.
Dazu soll eine Nachrichten-orientierte Middleware zum Einsatz kommen, die das Publish/Subscribe Paradigma unterstützt.
Da MQTT der de-facto Standard bei IoT Anwendungen ist, soll MQTT genutzt werden.
Ein vorhandener Broker soll dabei dahingehend verändert werden, dass sich Clients sowohl lokal als auch entfernt in der Cloud auf bestimmte Datenströme subskribieren können.
Der Broker sorgt dafür, dass Subskriptionen wenn möglich lokal bedient werden und Sensor-Daten nur dann in die Cloud geschickt werden, wenn das auch nötig ist.

\section{Umgebung}

Das Produkt soll weitgehend unabhängig von der genutzten Hard- und Software entwickelt werden.
Die folgende Produktumgebung stellt jedoch die Umgebung dar, die für eine Referenzimplementierung genutzt wird.

\subsection{Hardware}

\begin{description}
\item[relayr Wunderbar]
Das Sensor Starter Kit "Wunderbar" der Firmen Mikroelektronica bzw. relayr stellt die Sensordaten zur Verfügung.
Das "Wunderbar" Kit besteht aus fünf Sensoren und einem Baustein an dem alle Sensoren mit Groove-Anschluss angeschlossen werden können.
Die mitgelieferten Sensoren sind:
Temperatur/Feuchtigkeitssensor auf Basis des HTU21D, einem Low-Power Digital-Sensor für relative Luftfeuchtigkeit und Temperatur;
Beschleunigungssensor/Gyroskop auf Basis des MPU-6500 Sechs-Achsen Gyro- und Accelerometers;
Licht-/Näherungssensor auf Basis des TCS37717 High-Sensitivity RGB-Farb- und Näherungssensors;
IR-Transmitter mit dem SFH4441 High-Power 950 nm Infrator-Emitter;
Mikrofon/Geräuschsensor.
Zudem ist ein Gateway-Knoten vorhanden, der die Daten in die Cloud weiterleitet.
Die Verbindung zwischen den Sensoren und dem Gateway-Knoten benutzt das Protokoll Bluetooth Low Energy (BLE), welches auch unter dem Namen Bluetooth Smart vermarktet wird.
Dazu wird ein Nordic nRF51822 Modul eingesetzt.
Der Gateway-Knoten verbindet sich mit WLAN im 2,4 Ghz Band zu einem vorhandenen WLAN-Netz, um über dieses Netz die Daten in die Cloud von relayr weiterzuleiten.
Dazu wird ein Gainspan GS1500M IEEE 802.11b/g/n WLAN-Modul benutzt.
\item[Gateway]
Als Gateway, auf dem lokal die Verarbeitung und Verbreitung von Sensor-Daten durchgeführt wird, kommen besonders eingebettete Systeme in Frage, die ohnehin in Wohnungen bzw. kleineren bis mittleren Firmen vorhanden sind, etwas WLAN-Router oder anderes Netzwerk-Equipment, auf dem ein eingebettetes Linux-Betriebsystem läuft.
Als Ersatz soll zunächst ein Raspberry Pi benutzt werden, der als Einplatinencomputer mit Linux-Betriebssystem ähnlich der zu erwartenden Hardware ist.
\item[Cloud]
Als "Cloud" kommt jeder Rechner in Frage, der das MQTT Protokoll beherrscht.
Im produktiven Einsatz wird ein Broker auf einer virtuellen Maschine in der Cloud genutzt.
Der Prototyp soll zunächst mittels der Cloud von relayr realisiert werden.
\item[Client]
Als Client kommen alle Internetfähigen Rechner in betracht.
Von Bedeutung sind hier vor allem stationäre Rechner mit Internet-Browser und Mobiltelefone mit nativer App.
\end{description}

\subsection{Software}

\begin{description}
\item[MQTT]
Message Queuing Telemetry Transport (MQTT) ist ein Maschine-zu-Maschine (M2M) bzw. Internet der Dinge Kommunikationsprotokoll.
Es wurde als extrem schlankes Event-basierendes Protokoll entwickelt.
Bei der Entwicklung lag der Fokus auf Netzwerkanbindungen, die begrenzte Bandbreite bieten bze. teuer oder instabil sind.
Daher eignet sich MQTT besonders gut für Applikationen der Internet der Dinge.
Da es sich als de-facto Standard im Internet der Dinge entwickelt hat, müssen alle Komponenten mit MQTT kommunizieren können.
\end{description}

\section{Anforderungen}
% Hier sollen die Anforderungen an das Projektergebnis beschrieben werden.
% Es soll zwischen unbedingt erforderlichen Muss-Anforderungen und optionalen Kann-Anforderungen unterschieden werden.

Die einzelnen Teile des Produktes müssen bestimmte Funktionen und Anforderungen erfüllen.
Dabei wird unterschieden zwischen verbindlichen und optionalen Anforderungen.
Die verbindlichen Anforderungen müssen von Ihnen erfüllt werden, die optionalen sollen freiwillig umgesetzt werden.
Auch weitere, über die hier angegebenen Funktionalitäten hinausgehenden Erweiterungen sind denkbar.
Aufgelistet sind jeweils nur die Anforderungen des Sub-Projektes, nicht die Anforderungen aus dem gesamten Projekt.

\subsection{Verbindliche Anforderungen}
% Hier soll eine vollständige und nachprüfbare Beschreibung einer Anforderung an das Projektergebnis beschrieben werden.

Die zu entwickelnde Software muss folgende Anforderungen erfüllen:

\begin{itemize}
\item Brokerage im lokalen Netzwerk; dazu nötige Implementierungen:
    \begin{itemize}
    \item Subscribe
    \item Publish
    \item Weiterleitung der Nachrichten
    \end{itemize}
\item Weiterleitung der Nachrichten an die Cloud:
    \begin{itemize}
    \item Advertisement-Nachricht zum Anmelden der Sensoren
    \item Datenbank mit Informationen über angeschlossene Sensoren
    \item Subscriptions von Cloud zu lokalen Broker weiterleiten
    \item Selektives Weiterleiten von Notifications an Cloud
    \end{itemize}
\item Nicht funktionale Anforderungen:
    \begin{itemize}
    \item Modularität der Implementierung zwecks Wartbarkeit
    \item Publish/Subscribe Protokoll soll einfach zu wechseln sein
    \end{itemize}
\end{itemize}

\subsection{Optionale Anforderungen}

Folgende Anforderungen sollen optional erfüllt werden:

\begin{itemize}
\item Sensor Registry per REST-Interface zugänglich machen:
    \begin{itemize}
    \item Die vorhandenen Informationen der Datenbank werden zugänglich gemacht, um z.B. eine Suche auf den Daten implementieren zu können.
    \end{itemize}
\end{itemize}

\section{Arbeitspakete}
% Hier sollen die einzelnen Arbeitspakete, in denen das Projekt abgearbeitet werden soll beschrieben werden.
% Zusätzlich zu den Arbeitspaketen sollen Meilensteine definiert werden.
% Dies sind Zeitpunkte zu denen ein Zwischenergebnis bei der Projektbearbeitung erreicht wurde und eine neue Phase beginnt.

Die aus den genannten Zielbestimmung abgeleiteten Aufgaben werden in diesem Kapitel in Arbeitspakete überführt.

\begin{description}
\item[1. Meilenstein]
Weiterleitung aller Publish- und Subscribe-Nachrichten zum Cloud-broker; zusätzlich lokales Brokering.
\item[2. Meilenstein]
Implementierung von Advertisement-Nachrichten und Speicherung in einer Datenbank.
\item[3. Meilenstein]
Selektive Weiterleitung von Publish- und Subscribe-Nachrichten anhand der Daten der Advertisement-Nachrichten.
\end{description}

\appendix

\section{Checkliste der Anforderungen}

\subsection{Verbindliche Anforderungen}

\begin{itemize}
\item Brokerage im lokalen Netzwerk; dazu nötige Implementierungen:
    \begin{itemize}
    \item[$\square$] Subscribe
    \item[$\square$] Publish
    \item[$\square$] Weiterleitung der Nachrichten
    \end{itemize}
\item Weiterleitung der Nachrichten an die Cloud:
    \begin{itemize}
    \item[$\square$] Advertisement-Nachricht zum Anmelden der Sensoren
    \item[$\square$] Datenbank mit Informationen über angeschlossene Sensoren
    \item[$\square$] Subscriptions von Cloud zu lokalen Broker weiterleiten
    \item[$\square$] Selektives Weiterleiten von Notifications an Cloud
    \end{itemize}
\item Nicht funktionale Anforderungen:
    \begin{itemize}
    \item[$\square$] Modularität der Implementierung zwecks Wartbarkeit
    \item[$\square$] Publish/Subscribe Protokoll soll einfach zu wechseln sein
    \end{itemize}
\end{itemize}

\subsection{Optionale Anforderungen}

\begin{itemize}
\item Sensor Registry per REST-Interface zugänglich machen:
    \begin{itemize}
    \item[$\square$] Die vorhandenen Informationen der Datenbank werden zugänglich gemacht, um z.B. eine Suche auf den Daten implementieren zu können.
    \end{itemize}
\end{itemize}

\end{document}
